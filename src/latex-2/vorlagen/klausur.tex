% Vorlage für LaTeX-Abgaben / Gedächtnisprotokolle
% Erstellt von Jules Kreuer (juleskreuer.eu)
% Veröffentlicht am 24.04.2020 auf juleskreuer.eu
% und ppi.fsi.uni-tuebingen.de
% ---
\documentclass[a4paper]{article}

% Packages
% ---
\usepackage{amsmath} % Advanced Math Typesetting
\usepackage[utf8]{inputenc} % Unicode support (Umlaute etc.)
\usepackage[ngerman]{babel} % Change hyphenation rules

\usepackage{amssymb}
\usepackage{hyperref} % Links
\usepackage{graphicx} % Vilder
\usepackage{listings} % Source code highlighting
\usepackage[inline]{enumitem}
\usepackage{fullpage} % weniger abstand zu den seiten
\usepackage{tabularx} % bessere Tabellen
\usepackage{pdfpages} % pdf einbinden mit \includepdf[pages={2}]{x.pdf}

%Code
\usepackage{color}
\usepackage{colortbl}
\usepackage{textcomp}
\definecolor{listinggray}{gray}{0.9}
\definecolor{lbcolor}{rgb}{0.95,0.95,0.95}
\lstset{
	backgroundcolor=\color{lbcolor},
	tabsize=4,
	rulecolor=,
	basicstyle=\fontsize{10}{10}, % Fontgröße erstes: Text, zweites: Zahlen
	upquote=true,
	aboveskip={1.5\baselineskip},
	columns=fixed,
	showstringspaces=false,
	extendedchars=true,
	breaklines=true,
	prebreak = \raisebox{0ex}[0ex][0ex]{\ensuremath{\hookleftarrow}},
	frame=single,
	showtabs=false,
	showspaces=false,
	showstringspaces=false,
	identifierstyle=\ttfamily,
	keywordstyle=\color[rgb]{0,0,1},
	commentstyle=\color[rgb]{0.133,0.545,0.133},
	stringstyle=\color[rgb]{0.627,0.126,0.941},
}

\begin{document}
	% Name des Prüfenden
	\author{bei Prof. Musterfrau\\}
	% Name der Vorlesung
	\title{\vspace{-2cm}Gedächtnisprotokoll\\Name der Vorlesung}
	\date{\today{}} 
	\maketitle{} % Generates title
	\vspace{-1cm}
	% Hier bitte alle vorhandenen Informationen eintragen.
	\begin{lstlisting}
	Klausur:       Haupt/Nachklausur, SS/WS, JAHR
	Pruefer:       Prof. Pruefername
	Datum:         Datum der Pruefung
	Zeit:          Zeit zum Schreiben
	Punkte:		   Anzahl der Punkte
	Hilfsmittel:   Erlaubte Hilfsmittel
	               Bsp: Taschenrechner, Cheat-Sheet
	Sprache:	   Erlaubte Sprachen
	Modul:		   ggf. Modulnummer bei uneindeutigen Namen
	\end{lstlisting}
	
	\section{Titel erste Aufgabe - n Punkte}
	\subsection{}
	Lorem ipsum dolor sit amet, consetetur sadipscing elitr,sed diam nonumy eirmod tempor invidunt ut labore et dolore magna aliquyam erat, sed diam voluptua.
	At vero eos et accusam et justo duo dolores et ea rebum.

	\subsection{}
	Lorem ipsum dolor sit amet, consetetur sadipscing elitr,sed diam nonumy eirmod tempor invidunt ut labore et dolore magna aliquyam erat, sed diam voluptua.
	At vero eos et accusam et justo duo dolores et ea rebum.
\end{document}
