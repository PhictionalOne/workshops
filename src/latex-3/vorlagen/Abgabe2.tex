% ----------------------- TODO ---------------------------
% Diese Daten müssen pro Blatt angepasst werden:
\newcommand{\NUMBER}{1}
\newcommand{\EXERCISES}{4}
% Diese Daten müssen einmalig pro Vorlesung angepasst werden:
\newcommand{\COURSE}{Introduction to Statistical Machine Learning}
\newcommand{\STUDENTA}{Jules Kreuer}
\newcommand{\STUDENTB}{Meret Häusler}
% ----------------------- TODO ---------------------------

%Template 
\documentclass[a4paper]{scrartcl}
\usepackage[utf8]{inputenc}
\usepackage[ngerman]{babel}
\usepackage{geometry,forloop,fancyhdr,fancybox,lastpage}
\geometry{a4paper,left=3cm, right=3cm, top=3cm, bottom=3cm}

%Math
\usepackage{amsmath,amssymb,tabularx}

%Figures
\usepackage{graphicx,tikz,color,float}
\usetikzlibrary{shapes,trees}

%Algorithms
%\usepackage[ruled,linesnumbered]{algorithm2e}

%Kopf- und Fußzeile
\pagestyle {fancy}
\fancyhead[L]{\COURSE}
\fancyhead[R]{\today}

\fancyfoot[L]{}
\fancyfoot[C]{}
\fancyfoot[R]{Seite \thepage}

%Formatierung der Überschrift, hier nichts ändern
\def\header#1{
  \begin{center}
    {\Large \textbf{ Assignment #1}}\\
  \end{center}
}

%Definition der Punktetabelle, hier nichts ändern
\newcounter{punktelistectr}
\newcounter{punkte}
\newcommand{\punkteliste}[2]{%
  \setcounter{punkte}{#2}%
  \addtocounter{punkte}{-#1}%
  \stepcounter{punkte}%<-- also punkte = m-n+1 = Anzahl Spalten[1]
  \begin{center}%
  \begin{tabularx}{\linewidth}[]{@{}*{\thepunkte}{>{\centering\arraybackslash} X|}@{}>{\centering\arraybackslash}X}
      \forloop{punktelistectr}{#1}{\value{punktelistectr} < #2 } %
      {%
        \thepunktelistectr &
      }
      #2 &  $\Sigma$ \\
      \hline
      \forloop{punktelistectr}{#1}{\value{punktelistectr} < #2 } %
      {%
        &
      } &\\
      \forloop{punktelistectr}{#1}{\value{punktelistectr} < #2 } %
      {%
        &
      } &\\
    \end{tabularx}
  \end{center}
}

\begin{document}
\header{\NUMBER}

\begin{tabularx}{\linewidth}{m{0.2 \linewidth}X}
  \begin{minipage}{\linewidth}
    \STUDENTA\\
    \STUDENTB
  \end{minipage} & \begin{minipage}{\linewidth}
    \punkteliste{1}{\EXERCISES}
  \end{minipage}\\
\end{tabularx}
% ----------------------- TODO ---------------------------
% Hier werden die Aufgaben/Lösungen eingetragen:

\section*{Aufgabe 1}
\subsection*{a)}

Dies ist eine Aufgabe. Text könnt ihr hier einfach ganz normal schreiben, jedoch bricht
Latex
nicht
die
Zeilen
um, wenn ihr im Editor eine neue Zeile anfangt.
Um einen Zeilenumbruch \\
zu erhalten \\
verwendet zwei Backslash.\\
Einen neuen Absatz erhaltet ihr, wenn ihr zwei Zeilenumbrüche einfügt.

Es gibt zwei Arten des Mathemodus, einen im Text $x_1 = 5 \cdot 10^{42}$ und einen freistehenden.

$$x_2 = \sqrt{ \sin{ \alpha \times \beta} \cdot \frac{a}{b}} + \log_2{64}$$

Wenn der Mathemodus sich über mehrere Zeilen erstrecken soll, so könnt ihr die Umgebung align benutzen. Der Stern sagt hierbei, dass die Zeilen nicht nummeriert werden sollen. Das \& ist ein align-parameter, die Zeilen werden hierbei so ausgerichtet, dass die \& immer über einander stehen.

\begin{align*}
\text{Text im Mathemodus} & = 5\\
x_2 &= 7\\
\Sigma + \phi + \epsilon + w &= \text{sinnlose Zeichen}
\end{align*}


Ein Bild kann man folgendermaßen einbinden: \\
Entfernt dabei das \% und ersetzt Bild.jpg mit dem Dateinamen. \% kommentiert eine Zeile aus.

\begin{figure}[H]
\centering
%\includegraphics[width=11cm]{Bild.jpg}
\caption{Dies ist die Bildunterschrift}
\end{figure}

Eine Tabelle könnt ihr so anlegen:\\
In der zweiten geschweiften Klammer wird das Layout der Tabelle definiert. Das \& trennt hierbei die Spalten.\\
\ \\ %Trick um eine leere Zeile zu erhalten, gleichzeitig wird in der nächsten Zeile kein neuer Absatz begonnen
\begin{tabular}{c|lr}
  zentrierte Spalte & linksbündig & rechtsbündig\\
  \hline
  zweite Zeile & \\
\end{tabular}\\
\ \\
Eine neue Seite wird durch $\backslash$newpage erzwungen.
\newpage


Pseudocode kann mit dem Package algorithm2e gesetzt werden. Eine Dokumentation aller verfügbaren Kommandos ist im Internet leicht zu finden.


\vspace{1cm}

Binärbäume, AVL-Bäume, etc: \\

\begin{figure}[h!]
\centering
\ovalbox{
\begin{tikzpicture}[level/.style={sibling distance=60mm/#1}]
\node [circle,draw] (z){$n$}
  child {node [circle,draw] (a) {$\frac{n}{2}$}
    child {node [circle,draw] (b) {$\frac{n}{2^2}$}}
    child {node [circle,draw] (g) {$\frac{n}{2^2}$}}
  }
  child {node [circle,draw] (j) {$\frac{n}{2}$}
    child {node [circle,draw] (k) {$\frac{n}{2^2}$}}
  child {node [circle,draw] (l) {$\frac{n}{2^2}$}}
};

\end{tikzpicture}}\\

\caption{AVL-Baum nach Linksrotation um $\frac{n}{2}$}
\end{figure}

\vspace{1cm}

B-Bäume: \\


\end{document}
%%% Local Variables:
%%% mode: latex
%%% TeX-master: t
%%% End:
